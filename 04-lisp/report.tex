\documentclass[a4paper, 12pt]{extarticle}
\usepackage{polyglossia}
\setdefaultlanguage{ukrainian}
\setotherlanguage{english}

\usepackage{fontspec}
\setmainfont{CMU Sans Serif}
\newfontfamily{\cyrillicfonttt}{CMU Typewriter Text}

\usepackage[left=2.5cm, top=2.5cm, right=1.5cm, bottom=1.5cm]{geometry}

\linespread{1.5}
\setlength{\parindent}{0cm}

\usepackage{listings, color}
\definecolor{dkgreen}{rgb}{0,0.6,0}
\definecolor{gray}{rgb}{0.5,0.5,0.5}
\definecolor{mauve}{rgb}{0.58,0,0.82}
\lstset{frame=tb,
  language=Lisp,
  aboveskip=3mm,
  belowskip=3mm,
  inputencoding=utf8,
  extendedchars=\true,
  showstringspaces=false,
  columns=flexible,
  basicstyle={\small\ttfamily},
  numbers=left,
  numberstyle=\tiny\color{gray},
  keywordstyle=\color{blue},
  commentstyle=\color{dkgreen},
  stringstyle=\color{mauve},
  breaklines=true,
  breakatwhitespace=true,
  tabsize=2,
  lineskip={-1.5pt}
}

\usepackage{hyperref}
\usepackage{float}

\begin{document}

\begin{titlepage}
  \newgeometry{left=2.5cm, top=1.5cm, right=1.5cm, bottom=1.5cm}
  \thispagestyle{empty}\center
  \textsc{\uppercase{Міністерство освіти і науки України\\Національний технічний університет України\\<<Київський політехнічний інститут>>}}\\[1cm]
  \textsc{Кафедра обчислювальної техніки}\\[0.5cm]
  \vfill
  \textsc{\Large \textbf{Лабораторна робота \#4}}\\
  \textsc{з дисципліни}\\
  \textsc{\large ``Декларативне програмування''}\\
  \textsc{на тему:}\\
  \textsc{\Large <<Робота з динамічними базами даних>>}\\
  \vspace{3cm}
  \begin{minipage}{0.4\textwidth}
    \begin{flushleft} \large
      \emph{Виконав:}\\
      студент гр. ІП-32 \\
      \textsc{Ковальчук О. М.} % Your name
    \end{flushleft}
  \end{minipage}
  \begin{minipage}{0.4\textwidth}
    \begin{flushright} \large
      \emph{Перевірив:} \\
      доцент каф. АСОІУ \\
      \textsc{Баклан І. В.} % Supervisor's Name
    \end{flushright}
  \end{minipage}\\[4cm]
  \vfill

  Київ 2015
\end{titlepage}
\setcounter{page}{2}
\section{Мета і задачі}
\textbf{Метою роботи} є вивчення можливостей Lisp по організації динамічних баз даних.
\section{Визначення завдання}
\subsection{Визначення вихідних даних та індивідуального завдання}
Номер за списком --- 13. Отже, номер варіанту завдання --- 13.

\subsection{Завдання 1}
Написати програму, яка забезпечує створення бази даних та роботу із нею.
Об'єкти для збереження в БД: \url{http://www.kometa53.ru/upload/file/trans\%2001_11_2011.pdf}

Програма повинна надавати такі можливості:
\begin{itemize}
  \item створення БД;
  \item додавання інформації у БД;
  \item редагування інформації;
  \item запис БД на диск;
  \item вивантаження БД в оперативну пам'ять;
  \item перегляд інформації;
  \item видалення інформації з БД;
  \item пошук інформації в БД;
  \item сортування інформації;
\end{itemize}

\section{Виконання}
\lstinputlisting[caption=Завдання 1]{db-transformators.lsp}
\begin{lstlisting}
; insert
(db-insert (make-item "ТПК-25" 66 56 46 58 40 "4 отв, диаметр 3.2" 25 0.6 "А33"))
(db-insert (make-item "ТПК-40" 72 61 50 50 50 "4 отв, диаметр 4.5" 40 0.7 "В43"))

; select
(db-select-all) ; all
(db-select-one "ТПК-25") ; by type

; delete
(db-delete "ТПК-25") ; by type

; update
(db-update (make-item "ТПК-25" 98 57 36 68 30 "2 отв, диаметр 3.2" 25 0.6 "А33"))

; sorting
(db-sort ':type) ; by type
(db-sort ':l-mm) ; by l-mm

; I/O
(db-save "my-db.txt") ; write to disk
(db-load "my-db.txt") ; read from disk
\end{lstlisting}

\section{Висновки}

Під час виконання даної лабораторної роботи я навчився створювати найпростіші динамічні "Бази даних", здійснювати читання та запис у файл мовою \textit{Lisp} та застосував отримані навички для виконання поставлених завдань.

Вихідні коди виконаних завдань та даного звіту є у відкритому доступі та доступні за посиланням: \url{https://github.com/anxolerd/kpi-declarative-programming}

\end{document}
